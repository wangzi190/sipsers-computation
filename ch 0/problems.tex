\documentclass{article}

\usepackage{amsfonts}
\usepackage{amsmath}
\usepackage{amssymb}

\begin{document}
\title{Chapter 0 Problems}
\author{Hui}
\date{}
\maketitle

\section{Problem 0.10}
You cannot divide either side of the equation by $(a - b)$ because you assume $a = b$, meaning $a - b = 0$.

\section{Problem 0.11}
a) \textbf{Proof:} \textit{Basis:} Let $n = 1$. Then, $S(1) = 1$ and $\frac{1}{2}(1)(1 + 1) = 1$, which means $S(n) = \frac{1}{2}n(n+1)$ is true for $n = 1$.
\\\textit{Induction step:} Assume n is an arbitrary but fixed natural number such that $1 + 2 + ... + n = \frac{1}{2}n(n+1)$.
Then, $1 + 2 + ... + n + (n + 1) = \frac{1}{2}n(n+1) + (n + 1) = \frac{1}{2}n^2 + \frac{3}{2}n + 1 = \frac{1}{2}(n^2 + 3n + 2) = \frac{1}{2}(n + 1)((n + 1) + 1)$.
Since $1 + 2 + ... + n + (n + 1) = \frac{1}{2}(n + 1)((n + 1) + 1)$, $1 + 2 + ... + n = \frac{1}{2}n(n+1)$ is true for $n + 1$.
Thus, $S(n) = \frac{1}{2}n(n+1)$ for all natural numbers. Q.E.D.
\newline b) \textbf{Proof:} \textit{Basis:} Let $n = 1$. Then, $C(1) = 1$ and $\frac{1}{4}(1)^2(1 + 1)^2 = 1$, which means $C(n) = \frac{1}{4}n^2(n + 1)^2$ is true for $n = 1$.
\\\textit{Induction step:} Assume n is an arbitrary but fixed natural number such that $1^3 + 2^3 + ... + n^3 = \frac{1}{4}n^2(n+1)^2$.
Then, $1^3 + 2^3 + ... + n^3 + (n + 1)^3 = \frac{1}{4}n^2(n+1)^2 + (n + 1)^3 = (n + 1)^2(\frac{1}{4}n^2 + (n + 1)) = \frac{1}{4}(n + 1)^2(n^2 + 4n + 4) = \frac{1}{4}(n + 1)^2((n + 1) + 1)$.
Since $1^3 + 2^3 + ... + n^3 + (n + 1)^3 = \frac{1}{4}(n + 1)^2((n + 1) + 1)$, $1^3 + 2^3 + ... + n^3 = \frac{1}{4}n^2(n+1)^2$ is true for $n + 1$.
Thus, $C(n) = \frac{1}{4}n^2(n + 1)^2$ for all natural numbers. Q.E.D.
\newline \textbf{Conclusion:} Because $C(n) = \frac{1}{4}n^2(n + 1)^2$ and $S(n) = \frac{1}{2}n(n+1)$ for all natural numbers, $C(n) = S^2(n)$ for all natural numbers.
\section{Problem 0.12}

\section{Problem 0.13}

\section{Problem 0.14}

\section{Problem 0.15}

\end{document}