\documentclass{article}

\usepackage{amsfonts}
\usepackage{amsmath}
\usepackage{amssymb}

\begin{document}
\title{Chapter 0 Exercises}
\author{Hui}
\date{}
\maketitle

\section{Exercise 0.1}
a) The set of all odd natural numbers.
\\b) The set of all integers that are multiples of 2.
\\c) The set of all natural numbers divisible by 2.
\\d) The set of all natural numbers divisible by 2 and 3.
\\e) The set of all strings of 0s and 1s that are palindromic.
\\f) An empty set.

\section{Exercise 0.2}
a) $\{ n | n = 10^m$ for some $m\in \mathbb N\}$
\\b) $\{ n\in \mathbb Z | n > 5\}$
\\c) $\{ n\in \mathbb N | n < 5\}$
\\d) $\{$aba$\}$
\\e) $\{\epsilon\}$
\\f) $\emptyset$

\section{Exercise 0.3}
a) $A$ is not a subset of $B$.
\\b) $B$ is a subset of $A$.
\\c) $A \cup B = \{$x, y, z$\}$
\\d) $A \cap B = \{$x, y$\}$
\\e) $A \times B = \{($x, x$), ($x, y$), ($y, x$), ($y, y$), ($z, x$), ($z, y$)\}$
\\f) $\mathcal{P}(B) = \{\emptyset, \{$x$\}, \{$y$\}, B\}$

\section{Exercise 0.4}
The Cartesian product of $n$ sets is the set of all $n$-tuples where the first element is an element of the first operand, the second element is an element of the second operand, and so on. In the case of $A \times B$, the first element of each 2-tuple is an element of A, and the second element of each 2-tuple is an element of B. So, if $A$ has $a$ elements and $B$ has $b$ elements, to make every possible pair, you would need $a$ elements $b$ times, or $a * b$ pairs. Similarly, if $A$ has $a$ elements, $B$ has $b$ elements, and $C$ has $c$ elements, $A \times B \times C$ has $a * b * c$ elements.

\section{Exercise 0.5}
Each element of $\mathcal{P}(C)$ is a subset of $C$, meaning each element of $\mathcal{P}(C)$ is a permutation where each element of $C$ is either present or not present (2 possibilities per element). Thus, if $C$ is a set with $c$ elements, there are $2^c$ elements in $\mathcal{P}(C)$.

\section{Exercise 0.6}
a) $f(2) = 7$
\\b) $f:[1, 5] \rightarrow \{6, 7\}$
\\c) $g(2, 10) = 6$
\\d) $g:[1, 5] \times [6, 10] \rightarrow [6, 10]$
\\e) $g(4, f(4)) = g(4, 7) = 8$

\section{Exercise 0.7}
Let $A = \{1, 2, 3\}$.
\\a) $R_a = \{(1, 1), (2, 2), (3, 3), (1, 2), (2, 1), (2, 3), (3, 2)\}$
\\$R_a$ is reflexive because $\forall x\in A (x R_a x)$,
$R_a$ is symmetric because $\forall x\in A (x R_a y \rightarrow y R_a x)$,
and $R_a$ is not transitive because $\exists x\in A ((x R_a y \wedge y R_a z) \wedge \neg(x R_a z))$
\\b) $R_b = \{(1, 1), (2, 2), (3, 3), (1, 2), (2, 3), (1, 3)\}$
\\$R_b$ is reflexive because $\forall x\in A (x R_b x)$,
$R_b$ is transitive because $\forall x\in A ((x R_b y \wedge y R_b z) \rightarrow x R_b z)$,
and $R_b$ is not symmetric because $\exists x\in A (x R_b y \wedge \neg(y R_b x))$
\\c) $R_c = \{(1, 2), (2, 1), (1, 3), (3, 1), (2, 3), (3, 2)\}$
\\$R_c$ is symmetric because $\forall x\in A (x R_c y \rightarrow y R_c x)$,
$R_c$ is transitive because $\forall x\in A ((x R_c y \wedge y R_c z) \rightarrow x R_c z)$,
and $R_c$ is not reflexive because $\exists x\in A \neg(x R_c x)$

\section{Exercise 0.8}
Hey Alexa how to use tikz-automata

\section{Exercise 0.9}
$(\{1, 2, 3, 4, 5, 6\}, \{(1, 4), (1, 5), (1, 6), (2, 4), (2, 5), (2, 6), (3, 4), (3, 5), (3, 6)\})$

% \begin{align*}
% \end{align*}

\end{document}